\documentclass[Journal,SectionNumbers,SingleSpace]{ascelike}

\usepackage{subfigure}
\usepackage{epsfig}
%\usepackage{timesmt}

\usepackage[utf8]{inputenc}
\usepackage[OT1]{fontenc}
\usepackage{graphicx}
\usepackage[english]{babel}

\usepackage{amsmath}
\usepackage{amsfonts}
\usepackage{amssymb}
\usepackage{amsthm}
\usepackage{bm}

\usepackage[usenames,dvipsnames]{xcolor}
\usepackage{url}
\usepackage{tikz}
\usetikzlibrary{shapes.misc,fit}

%\usepackage[bookmarks]{hyperref}
\usepackage[colorlinks=true,citecolor=red,linkcolor=black]{hyperref}

\newcommand{\reals}{\mathbb{R}}
\newcommand{\posreals}{\reals_{>0}}
\newcommand{\posrealszero}{\reals_{\ge 0}}
\newcommand{\naturals}{\mathbb{N}}

\newcommand{\dd}{\,\mathrm{d}}

\newcommand{\mbf}[1]{\mathbf{#1}}
\newcommand{\bs}[1]{\boldsymbol{#1}}
\renewcommand{\vec}[1]{{\bm#1}}

\newcommand{\uz}{^{(0)}} % upper zero
\newcommand{\un}{^{(n)}} % upper n
\newcommand{\ui}{^{(i)}} % upper i

\newcommand{\ul}[1]{\underline{#1}}
\newcommand{\ol}[1]{\overline{#1}}

\newcommand{\Rsys}{R_\text{sys}}
\newcommand{\lRsys}{\ul{R}_\text{sys}}
\newcommand{\uRsys}{\ol{R}_\text{sys}}

\newcommand{\Fsys}{F_\text{sys}}
\newcommand{\lFsys}{\ul{F}_\text{sys}}
\newcommand{\uFsys}{\ol{F}_\text{sys}}

\def\Tsys{T_\text{sys}}

\newcommand{\E}{\operatorname{E}}
\newcommand{\V}{\operatorname{Var}}
\newcommand{\wei}{\operatorname{Wei}} % Weibull Distribution
\newcommand{\ig}{\operatorname{IG}}   % Inverse Gamma Distribution

\def\yz{y\uz}
\def\yn{y\un}
%\def\yi{y\ui}
\newcommand{\yfun}[1]{y^{({#1})}}
\newcommand{\yfunl}[1]{\ul{y}^{({#1})}}
\newcommand{\yfunu}[1]{\ol{y}^{({#1})}}

\def\ykz{y\uz_k}
\def\ykn{y\un_k}

\def\yzl{\ul{y}\uz}
\def\yzu{\ol{y}\uz}
\def\ynl{\ul{y}\un}
\def\ynu{\ol{y}\un}
\def\yil{\ul{y}\ui}
\def\yiu{\ol{y}\ui}

\def\ykzl{\ul{y}\uz_k}
\def\ykzu{\ol{y}\uz_k}
\def\yknl{\ul{y}\un_k}
\def\yknu{\ol{y}\un_k}


\def\nz{n\uz}
\def\nn{n\un}
%\def\ni{n\ui}
\newcommand{\nfun}[1]{n^{({#1})}}
\newcommand{\nfunl}[1]{\ul{n}^{({#1})}}
\newcommand{\nfunu}[1]{\ol{n}^{({#1})}}

\def\nkz{n\uz_k}
\def\nkn{n\un_k}
\newcommand{\nkzfun}[1]{n\uz_{#1}}


\def\nzl{\ul{n}\uz}
\def\nzu{\ol{n}\uz}
\def\nnl{\ul{n}\un}
\def\nnu{\ol{n}\un}
\def\nil{\ul{n}\ui}
\def\niu{\ol{n}\ui}

\def\nkzl{\ul{n}\uz_k}
\def\nkzu{\ol{n}\uz_k}
\def\nknl{\ul{n}\un_k}
\def\nknu{\ol{n}\un_k}


\def\taut{\tau(\vec{t})}
\def\ttau{\tilde{\tau}}
\def\ttaut{\ttau(\vec{t})}

\def\MZ{\mathcal{M}\uz}
\def\MN{\mathcal{M}\un}

\def\MkZ{\mathcal{M}\uz_k}
\def\MkN{\mathcal{M}\un_k}

\def\PkZ{\Pi\uz_k}
\def\PkN{\Pi\un_k}
\newcommand{\PZi}[1]{\Pi\uz_{#1}}


\def\tnow{t_\text{now}}
\def\tpnow{t^+_\text{now}}

\newcommand{\comments}[1]{{\small\color{gray} #1}}

\newtheorem{example}{Example}

\allowdisplaybreaks

\title{Notes for ISIPTA poster}
\author{Gero Walter, Frank P.A. Coolen, Simme Douwe Flapper}

\begin{document}
\title{ROBUST REMAINING USEFUL LIFE FOR COMPLEX SYSTEMS\\ or\\ ROBUST BAYESIAN RELIABILITY FOR COMPLEX SYSTEMS\\ or \ldots}

\author{
Gero Walter%
\thanks{
School of Industrial Engineering,
Eindhoven University of Technology, Eindhoven, The Netherlands.
E-mail: g.walter@tue.nl.},
\ Ph.D.
\\
and
Frank P.A. Coolen%
\thanks{
Department of Mathematical Sciences,
Durham University, Durham, United Kingdom.
E-mail: frank.coolen@durham.ac.uk.},
\ Ph.D.
}

\maketitle

% slightly adapted ESREL abstract
\begin{abstract}
In reliability engineering, data about failure events is often scarce.
To arrive at meaningful estimates for the reliability of a system,
it is therefore often necessary to also include expert information in the analysis,
which is straightforward in the Bayesian approach by using an informative prior distribution.
%
A problem that then can arise is called prior-data conflict:
from the viewpoint of the prior, the observed data seem very surprising,
i.e., the information from data is in conflict with the prior assumptions.
It has been recognised that models based on conjugate priors can be insensitive to prior-data conflict,
in the sense that the spread of the posterior distribution does not increase in case of such a conflict,
thus conveying a false sense of certainty by communicating that we can quantify the reliability of a system quite precisely when in fact we cannot.
%
We present an approach to mitigate this issue, by considering sets of prior distributions
to model vague knowledge on component lifetimes,
and study how surprisingly early or late component failures
affect the prediction of the reliability of a system with arbitrary layout,
making use of survival signature to characterise the system under study.
Our approach can be seen as a robust Bayesian procedure or imprecise probability method
that appropriately reflects surprising data in the posterior system survival function or other posterior inferences.
\end{abstract}

\KeyWords{System Reliability, Imprecise Probability, Survival Signature, Robust Bayesian, Remaining Useful Life}

***slightly adapted ESREL abstract, must be 150--175 words long (this has 226), may not contain references or mathematics.

\section{Introduction}

***reuse ESREL paper intro

\section{The Meat}

System with components of $k=1,\ldots,K$ different types.
There are $n_k$ components of type $k$ in the system.

For each $k$, component lifetimes ($i=1,\ldots,n_k$) are $T_i^k \mid \lambda_k \sim \wei(\kappa,\lambda_k)$,
where $\kappa$ is fixed and known:
\begin{linenomath*}
\begin{align}
f_k(t_i^k \mid \lambda_k) &= \frac{\kappa}{\lambda_k} (t_i^k)^{\kappa-1} e^{-\frac{(t_i^k)^{\kappa-1}}{\lambda_k}} \\
F_k(t_i^k \mid \lambda_k) &= 1 - e^{-\frac{(t_i^k)^\kappa}{\lambda_k}} = P_k(T_i^k \leq t_i^k \mid \lambda_k)
\end{align}
\end{linenomath*}

Conjugate prior is $\lambda_k \sim \ig(\alpha_k,\beta_k)$:
\begin{linenomath*}
\begin{align}
f_{\lambda_k}(\lambda_k\mid \alpha_k,\beta_k) &= \frac{(\beta_k)^{\alpha_k}}{\Gamma(\alpha_k)} \lambda_k^{-\alpha_k -1} e^{-\frac{\beta_k}{\lambda_k}}
\end{align}
\end{linenomath*}
In terms of canonical parameters $\nkz$ and $\ykz$, we assume $\lambda_k \mid \nkz,\ykz \sim \ig(\nkz + 1, \nkz\ykz)$.

Set of priors $\MkZ$ defined by $(\nkz,\ykz) \in \PkZ = [\nkzl,\nkzu] \times [\ykzl,\ykzu]$.

Observing a one of a kind system with $n_k$ components of type $k$ running until $\tnow$,
where $e_k$ components have failed by $\tnow$, and $n_k - e_k$ components still function:
\begin{linenomath*}
\begin{align}
\mbf{t}^k_{e_k;n_k} &= \big( \underbrace{t^k_1, \ldots, t^k_{e_k}}_{e_k \text{failure times}},
                             \underbrace{\tpnow, \ldots, \tpnow}_{n_k-e_k \text{censored obs.}} \big)
\end{align}
\end{linenomath*}


\end{document}
