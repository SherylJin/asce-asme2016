\documentclass[Journal,SectionNumbers,SingleSpace]{ascelike}

\usepackage{subfigure}
\usepackage{epsfig}
%\usepackage{timesmt}

\usepackage[utf8]{inputenc}
\usepackage[OT1]{fontenc}
\usepackage{graphicx}
\usepackage[english]{babel}

\usepackage{amsmath}
\usepackage{amsfonts}
\usepackage{amssymb}
\usepackage{amsthm}
\usepackage{bm}

\usepackage[usenames,dvipsnames]{xcolor}
\usepackage{url}
\usepackage{tikz}
\usetikzlibrary{shapes.misc,fit}

%\usepackage[bookmarks]{hyperref}
\usepackage[colorlinks=true,citecolor=red,linkcolor=black]{hyperref}

\newcommand{\reals}{\mathbb{R}}
\newcommand{\posreals}{\reals_{>0}}
\newcommand{\posrealszero}{\reals_{\ge 0}}
\newcommand{\naturals}{\mathbb{N}}

\newcommand{\dd}{\,\mathrm{d}}

\newcommand{\mbf}[1]{\mathbf{#1}}
\newcommand{\bs}[1]{\boldsymbol{#1}}
\renewcommand{\vec}[1]{{\bm#1}}

\newcommand{\uz}{^{(0)}} % upper zero
\newcommand{\un}{^{(n)}} % upper n
\newcommand{\ui}{^{(i)}} % upper i

\newcommand{\ul}[1]{\underline{#1}}
\newcommand{\ol}[1]{\overline{#1}}

\newcommand{\Rsys}{R_\text{sys}}
\newcommand{\lRsys}{\ul{R}_\text{sys}}
\newcommand{\uRsys}{\ol{R}_\text{sys}}

\newcommand{\Fsys}{F_\text{sys}}
\newcommand{\lFsys}{\ul{F}_\text{sys}}
\newcommand{\uFsys}{\ol{F}_\text{sys}}

\def\Tsys{T_\text{sys}}

\newcommand{\E}{\operatorname{E}}
\newcommand{\V}{\operatorname{Var}}
\newcommand{\wei}{\operatorname{Wei}} % Weibull Distribution
\newcommand{\ig}{\operatorname{IG}}   % Inverse Gamma Distribution

\def\yz{y\uz}
\def\yn{y\un}
%\def\yi{y\ui}
\newcommand{\yfun}[1]{y^{({#1})}}
\newcommand{\yfunl}[1]{\ul{y}^{({#1})}}
\newcommand{\yfunu}[1]{\ol{y}^{({#1})}}

\def\ykz{y\uz_k}
\def\ykn{y\un_k}

\def\yzl{\ul{y}\uz}
\def\yzu{\ol{y}\uz}
\def\ynl{\ul{y}\un}
\def\ynu{\ol{y}\un}
\def\yil{\ul{y}\ui}
\def\yiu{\ol{y}\ui}

\def\ykzl{\ul{y}\uz_k}
\def\ykzu{\ol{y}\uz_k}
\def\yknl{\ul{y}\un_k}
\def\yknu{\ol{y}\un_k}


\def\nz{n\uz}
\def\nn{n\un}
%\def\ni{n\ui}
\newcommand{\nfun}[1]{n^{({#1})}}
\newcommand{\nfunl}[1]{\ul{n}^{({#1})}}
\newcommand{\nfunu}[1]{\ol{n}^{({#1})}}

\def\nkz{n\uz_k}
\def\nkn{n\un_k}
\newcommand{\nkzfun}[1]{n\uz_{#1}}


\def\nzl{\ul{n}\uz}
\def\nzu{\ol{n}\uz}
\def\nnl{\ul{n}\un}
\def\nnu{\ol{n}\un}
\def\nil{\ul{n}\ui}
\def\niu{\ol{n}\ui}

\def\nkzl{\ul{n}\uz_k}
\def\nkzu{\ol{n}\uz_k}
\def\nknl{\ul{n}\un_k}
\def\nknu{\ol{n}\un_k}


\def\taut{\tau(\vec{t})}
\def\ttau{\tilde{\tau}}
\def\ttaut{\ttau(\vec{t})}

\def\MZ{\mathcal{M}\uz}
\def\MN{\mathcal{M}\un}

\def\MkZ{\mathcal{M}\uz_k}
\def\MkN{\mathcal{M}\un_k}

\def\PkZ{\Pi\uz_k}
\def\PkN{\Pi\un_k}
\newcommand{\PZi}[1]{\Pi\uz_{#1}}


\def\tnow{t_\text{now}}
\def\tpnow{t^+_\text{now}}

\newcommand{\comments}[1]{{\small\color{gray} #1}}

\newtheorem{example}{Example}

\allowdisplaybreaks

\title{Notes for ISIPTA poster}
\author{Gero Walter, Frank P.A. Coolen, Simme Douwe Flapper}

\begin{document}
\title{ROBUST REMAINING USEFUL LIFE FOR COMPLEX SYSTEMS\\ or\\ ROBUST BAYESIAN RELIABILITY FOR COMPLEX SYSTEMS\\ or \ldots}

\author{
Gero Walter%
\thanks{
School of Industrial Engineering,
Eindhoven University of Technology, Eindhoven, The Netherlands.
E-mail: g.walter@tue.nl.},
\ Ph.D.
\\
and
Frank P.A. Coolen%
\thanks{
Department of Mathematical Sciences,
Durham University, Durham, United Kingdom.
E-mail: frank.coolen@durham.ac.uk.},
\ Ph.D.
\\
and ???
}

\maketitle

Invited for special issue ``The treatment of uncertainty in risk and reliability modelling and decision making''
(special issue code SI022A), guest editors Luca Podolfillini, Bruno Sudret, Enrico Zio,
in ASCE-ASME Journal of Risk and Uncertainty in Engineering Systems: Part A (Civil Engineering)

***``For most ASCE journals, the maximum number of words and word-equivalents is 10,000 for technical papers''


% slightly adapted ESREL abstract
\begin{abstract}
In reliability engineering, data about failure events is often scarce.
To arrive at meaningful estimates for the reliability of a system,
it is therefore often necessary to also include expert information in the analysis,
which is straightforward in the Bayesian approach by using an informative prior distribution.
%
A problem that then can arise is called prior-data conflict:
from the viewpoint of the prior, the observed data seem very surprising,
i.e., the information from data is in conflict with the prior assumptions.
It has been recognised that models based on conjugate priors can be insensitive to prior-data conflict,
in the sense that the spread of the posterior distribution does not increase in case of such a conflict,
thus conveying a false sense of certainty by communicating that we can quantify the reliability of a system quite precisely when in fact we cannot.
%
We present an approach to mitigate this issue, by considering sets of prior distributions
to model vague knowledge on component lifetimes,
and study how surprisingly early or late component failures
affect the prediction of the reliability of a system with arbitrary layout,
making use of survival signature to characterise the system under study.
Our approach can be seen as a robust Bayesian procedure or imprecise probability method
that appropriately reflects surprising data in the posterior system survival function or other posterior inferences.
\end{abstract}

\KeyWords{System Reliability, Imprecise Probability, Survival Signature, Robust Bayesian, Remaining Useful Life}

***slightly adapted ESREL abstract, must be 150--175 words long (this has 226), may not contain references or mathematics.


\section{Introduction}

In reliability engineering, a central task is to describe the reliability of a complex system.
This is usually done by determining the \emph{reliability function} $R(t)$,
in other contexts also known as the \emph{survival function} $S(t)$,
giving the probability that the system has not failed by time $t$:
\begin{align}
R_\text{sys}(t) = P(T_\text{sys} \geq t)\,,
\end{align}
where $T_\text{sys}$ is the random variable giving the failure time of the system. %
%\footnote{}
Based on the distribution of $T_\text{sys}$, which can also be expressed
in terms of the cumulative distribution function $F_\text{sys}(t) = 1 - R_\text{sys}(t)$,
the density $p_\text{sys}(t)$ or the hazard rate $\lambda_\text{sys}(t)$,
decisions about, e.g., scheduling of maintenance work can be made.

Often, there is no failure data for the system itself
(e.g., if the system is a prototype, or the system is used under unique circumstances),
but there is some information about failure times for the components the system is made of.
Information on the distribution of component failure times $T_i$,
where $i = 1, \ldots, n$ if the system consists of $n$ components,
can then be used to derive the distribution of $T_\text{sys}$.
In this paper, we consider*** a simple parallel system
where all components are of the same type,
and our observations consist solely of the failure times of one or more components in this system.
%In the last section, we will briefly describe
%how the approach can be generalized to arbitrary system structures
%using the survival signature \cite{2012:survsign},
%which allows for different types of components and
%the inclusion of data from component tests, as in \citeNP{2014:bayessurvsign}.
%
We assume a parametric probability distribution for component lifetimes $T_i$. 
The Bayesian approach allows to base estimation of the component failure distributions
on both data and further knowledge not given by the data,
the latter usually provided in the form of expert knowledge.
This knowledge is encoded in form of a so-called prior distribution,
a distribution on the parameter(s) of the component lifetime distribution.
This expert knowledge is especially important when there is very few data on the components (like in our scenario),
as only with its help meaningful estimates for the system reliability can be made.


\section{Bayesian Analysis of Weibull Lifetimes}

System with components of $k=1,\ldots,K$ different types.
There are $n_k$ components of type $k$ in the system.

For each $k$, component lifetimes ($i=1,\ldots,n_k$) are $T_i^k \mid \lambda_k \sim \wei(\kappa,\lambda_k)$,
where $\kappa$ is fixed and known:
\begin{linenomath*}
\begin{align}
f_k(t_i^k \mid \lambda_k) &= \frac{\kappa}{\lambda_k} (t_i^k)^{\kappa-1} e^{-\frac{(t_i^k)^{\kappa-1}}{\lambda_k}} \\
F_k(t_i^k \mid \lambda_k) &= 1 - e^{-\frac{(t_i^k)^\kappa}{\lambda_k}} = P_k(T_i^k \leq t_i^k \mid \lambda_k)
\end{align}
\end{linenomath*}

Conjugate prior is $\lambda_k \sim \ig(\alpha_k,\beta_k)$:
\begin{linenomath*}
\begin{align}
f_{\lambda_k}(\lambda_k\mid \alpha_k,\beta_k) &= \frac{(\beta_k)^{\alpha_k}}{\Gamma(\alpha_k)} \lambda_k^{-\alpha_k -1} e^{-\frac{\beta_k}{\lambda_k}}
\end{align}
\end{linenomath*}
In terms of canonical parameters $\nkz$ and $\ykz$, we assume $\lambda_k \mid \nkz,\ykz \sim \ig(\nkz + 1, \nkz\ykz)$.


\section{Sets of Priors}

Set of priors $\MkZ$ defined by $(\nkz,\ykz) \in \PkZ = [\nkzl,\nkzu] \times [\ykzl,\ykzu]$.

***illustrate prior-data conflict similar to ESREL paper


\section{Robust RUL Estimation for Complex Systems via the Survival Signature}

\subsection{Right-censored Lifetimes}

Observing a one of a kind system with $n_k$ components of type $k$ running until $\tnow$,
where $e_k$ components have failed by $\tnow$, and $n_k - e_k$ components still function:
\begin{linenomath*}
\begin{align}
\mbf{t}^k_{e_k;n_k} &= \big( \underbrace{t^k_1, \ldots, t^k_{e_k}}_{e_k \text{failure times}},
                             \underbrace{\tpnow, \ldots, \tpnow}_{n_k-e_k \text{censored obs.}} \big)
\end{align}
\end{linenomath*}

***change to general right-censoring? So $n_k-e_k \to c_k$, $e_k+c_k=n_k$,
introduce sets of observed and censored?

\subsection{Bayesian Estimation of Component Scale Parameter}

Posterior:
\begin{linenomath*}
\begin{align}
f_{\lambda_k\mid\ldots}(\lambda_k\mid\nkz,\ykz,\mbf{t}^k_{e_k;n_k})
 &\propto f_{\lambda_k}(\lambda_k)
          \big[ 1- F_k(\tnow\mid\lambda_k) \big]^{n_k-e_k}
          \prod_{i=1}^{e_k} f_k(t_i^k \mid \lambda_k) 
\end{align}
\end{linenomath*}
and so $\lambda_k\mid\nkz,\ykz,\mbf{t}^k_{e_k;n_k} \sim \ig(\nkn + 1, \nkn\ykn)$, where
\begin{linenomath*}
\begin{align}
\nkn + 1 &= \nkz + e_k + 1 \\
\nkn\ykn &= \nkz\ykz + (n_k-e_k) \tnow^\kappa + \sum_{i=1}^{e_k} (t_i^k)^\kappa
\end{align}
\end{linenomath*}
so the posterior expectation for $\lambda$ is
\begin{linenomath*}
\begin{align}
\E[\lambda_k\mid\nkn,\ykn] &= \ykn
                            = \frac{\nkz}{\nkz + e_k} \ykz +
                              \frac{e_k}{\nkz + e_k} \cdot \Big[\frac{n_k-e_k}{e_k} \tnow^\kappa + \frac{1}{e_k}\sum_{i=1}^{e_k} (t_i^k)^\kappa \Big]
\end{align}
\end{linenomath*}

\subsection{RUL using the Survival Signature}

System survival calculated by %(see Risk Analysis paper eq. (7))
\begin{linenomath*}
\begin{align}
P(\Tsys > t) &= \sum_{l_1=0}^{n_1} \cdots \sum_{l_K=0}^{n_K} \Phi(l_1,\ldots,l_K) P\Big( \bigcap_{k=1}^K \{ C^k_t = l_k\} \Big) \\
\intertext{ and assuming components of different types are independent,}
             &= \sum_{l_1=0}^{n_1} \cdots \sum_{l_K=0}^{n_K} \Phi(l_1,\ldots,l_K) \prod_{k=1}^K P(C^k_t = l_k)
\end{align}
\end{linenomath*}
where the survival signature $\Phi(l_1,\ldots,l_K)$ is non-decreasing in each $l_k$ for coherent systems,
and $P(C^k_t = l_k)$ is the (predictive) probability that exactly $l_k$ components of type $k$ function at time $t$.

A priori, we have, for $l_k = 0,1,\ldots, n_k$,
\begin{linenomath*}
\begin{align}
P(C^k_t = l_k\mid\nkz,\ykz)
 &= { n_k \choose l_k} \int [F_k(t \mid\lambda_k)]^{n_k-l_k}
                        [1 - F_k(t \mid\lambda_k)]^{l_k} f_{\lambda_k}(\lambda_k\mid\nkz,\ykz) \dd \lambda_k
\end{align}
\end{linenomath*}
under the assumption that components of the same type are independent given $\lambda_k$.

\subsection{Posterior Predictive Distribution}

For the situation with the system observed until $\tnow$, with data $\mbf{t}^k_{e_k;n_k}$,
we need the posterior predictive probability $P(C^k_t = l_k\mid\nkz,\ykz, \mbf{t}^k_{e_k;n_k})$.

For $l_k > n_k - e_k$, we must have $P(C^k_t = l_k\mid\nkz,\ykz, \mbf{t}^k_{e_k;n_k}) = 0$,
as at most $n_k - e_k$ components can function for $t > \tnow$.

%(Should one work with the subsystem consisting only of the non-failed $n_k - e_k$ components?)\\

A posteriori, we have, for $l_k = 0,1,\ldots, n_k - e_k$,
\begin{linenomath*}
\begin{align}
P(C^k_t = l_k\mid\nkz,\ykz, \mbf{t}^k_{e_k;n_k})
 &= { n_k - e_k \choose l_k} \int [P_k(T \leq t \mid T > \tnow, \lambda_k)]^{n_k - e_k - l_k} \times \\ & \hspace*{2ex}
                              [1 - P_k(T \leq t \mid T > \tnow, \lambda_k)]^{l_k}
    f_{\lambda_k\mid\ldots}(\lambda_k\mid\nkz,\ykz,\mbf{t}^k_{e_k;n_k}) \dd \lambda_k
\end{align}
\end{linenomath*}

Now,
\begin{linenomath*}
\begin{align}
P_k(T \leq t \mid T > \tnow, \lambda_k)
 &= \frac{P_k(\tnow < T \leq t \mid\lambda_k)}{P_k(T > \tnow \mid \lambda_k)}
  = \frac{F_k(t\mid\lambda_k) - F_k(\tnow\mid\lambda_k)}{1-F_k(\tnow\mid\lambda_k)} \\
 &= \frac{e^{-\frac{(\tnow)^\kappa}{\lambda_k}} - e^{-\frac{t^\kappa}{\lambda_k}}}{e^{-\frac{(\tnow)^\kappa}{\lambda_k}}}
  = 1 - e^{-\frac{t^\kappa - (\tnow)^\kappa}{\lambda_k}}
\end{align}
\end{linenomath*}

With this and the posterior substituted in, we get
\begin{linenomath*}
\begin{align}
\lefteqn{P(C^k_t = l_k\mid\nkz,\ykz, \mbf{t}^k_{e_k;n_k})} \\
 &= { n_k - e_k \choose l_k} \int \Big[1 - e^{-\frac{t^\kappa - (\tnow)^\kappa}{\lambda_k}}\Big]^{n_k - e_k - l_k}
                                  \Big[    e^{-\frac{t^\kappa - (\tnow)^\kappa}{\lambda_k}}\Big]^{l_k} \times \\ & \hspace*{28ex}
    \frac{[\nkn\ykn]^{\nkn + 1}}{\Gamma(\nkn+1)} \lambda_k^{-(\nkn + 1) - 1} e^{-\frac{\nkn\ykn}{\lambda_k}} \dd \lambda_k \\
 &= { n_k - e_k \choose l_k} \sum_{j=0}^{n_k-e_k-l_k} (-1)^j { n_k - e_k - l_k \choose j} \frac{[\nkn\ykn]^{\nkn + 1}}{\Gamma(\nkn+1)} 
    \times \\ & \hspace*{25ex}
    \int \lambda_k^{-(\nkn + 1) - 1} e^{-\frac{(l_k + j) (t^\kappa - (\tnow)^\kappa) + \nkn\ykn}{\lambda_k}} \dd \lambda_k
\end{align}
\end{linenomath*}

The terms remaining under the integral form the core of an $\ig(\nkn + 1, \nkn\ykn + (l_k + j) (t^\kappa - (\tnow)^\kappa))$,
allowing us to solve the integral using the corresponding normalization constant.
\begin{linenomath*}
\begin{align}
\lefteqn{P(C^k_t = l_k\mid\nkz,\ykz, \mbf{t}^k_{e_k;n_k})} \\
 &= { n_k - e_k \choose l_k} \sum_{j=0}^{n_k-e_k-l_k} (-1)^j { n_k - e_k - l_k \choose j}
    \left(\frac{\nkn\ykn}{\nkn\ykn + (l_k + j) (t^\kappa - (\tnow)^\kappa)}\right)^{\nkn + 1} \\
 &= \sum_{j=0}^{n_k-e_k-l_k} (-1)^j \frac{(n_k - e_k)!}{l_k! j! (n_k - e_k - l_k - j)!}   
    \left(\frac{\nkn\ykn}{\nkn\ykn + (l_k + j) (t^\kappa - (\tnow)^\kappa)}\right)^{\nkn + 1} \\
 &= \sum_{j=0}^{n_k-e_k-l_k} (-1)^j \frac{(n_k - e_k)!}{l_k! j! (n_k - e_k - l_k - j)!} \times \\ & \hspace*{13ex}  
    \left(\frac{\nkz\ykz + \sum_{i=1}^{e_k} (t_i^k)^\kappa + (n_k-e_k) (\tnow)^\kappa }%
               {\nkz\ykz + \sum_{i=1}^{e_k} (t_i^k)^\kappa + (n_k-e_k-l_k-j) (\tnow)^\kappa + (l_k + j) t^\kappa }\right)^{%
    \nkz + e_k + 1} 
\end{align}
\end{linenomath*}
for $l_k \in \{0,1,\ldots,n_k-e_k\}$.

We actually need to compute only those $P(C^k_t = l_k\mid \ldots)$ with $l_k$'s for which $\Phi(l_1,\ldots,l_K) > 0$.

Seen as function in $t$, we see that $t$ appears $n_k-e_k-l_k+1$ times,
once in each summand, in the denominator of the fraction powered to the $\nkn+1$,
and summands whith even $j$ are decreasing in $t$,
while summands whith odd $j$ are increasing in $t$.
Nevertheless, in total $P(C^k_t = l_k\mid\nkz,\ykz, \mbf{t}^k_{e_k;n_k})$
must be decreasing in $t$, as the remaining $n_k-e_k$ components continue to age,
and so the probability that $l_k$ of them remain operational must decrease.

\subsection{Optimizing over Sets of Parameters}

%I would think that
%$\min / \max_{(\nkz,\ykz) \in \PkZ} P(C^k_t = l_k\mid\nkz,\ykz, \mbf{t}^k_{e_k;n_k})$
%must be solved by numeric optimization for each $k, l_k, t$ 
%because of the alternating signs for the summands.
%(In absolute terms, each summand is increasing in $\ykz$, but not monotone in $\nkz$.)

%(Box-constraint optimization possible, e.g., with option \texttt{L-BFGS-B} of \texttt{optim} in \textbf{R}.)
Similar to the considerations about $t$, as higher values for $\ykz$ mean higher expected lifetimes for the components,
the larger $\ykz$, the higher the probability that many components survive
($P(C^k_t = l_k\mid\ldots)$ high for large $l_k$)
and, with that, the lower the probability that few or no components survive
($P(C^k_t = l_k\mid\ldots)$ low for small $l_k$).

In terms of the cumulative probability mass function (cmf) for $C^k_t$, which can be written as 
\begin{linenomath*}
\begin{align}
%F_{C^k_t}(l_k \mid \nkz,\ykz, \mbf{t}^k_{e_k;n_k}) = P(C^k_t \leq l_k\mid\nkz,\ykz, \mbf{t}^k_{e_k;n_k}) 
% = \sum_{q=0}^{l_k} P(C^k_t = q\mid\nkz,\ykz, \mbf{t}^k_{e_k;n_k})
F_{C^k_t}(l_k \mid \ldots) = P(C^k_t \leq l_k\mid\ldots) = \sum_{q=0}^{l_k} P(C^k_t = q\mid\ldots)\,,
\end{align}
\end{linenomath*}
we therefore conclude that
the lower cmf $\ul{F}(C^k_t = l_k\mid \ldots)$ must be obtained with the lower bounds for all $\ykz$, $k=1,\ldots,K$,
and the upper cmf $\ol{F}(C^k_t = l_k\mid \ldots)$ must be obtained with the respective upper bounds.

***proove rigorously (would need a kind of likelihood difference order)?
%Do I get your comments right that based on lower and upper bounds
%$\ul{P}(C^k_t = l_k\mid \ldots)$ and $\ol{P}(C^k_t = l_k\mid \ldots)$ for all $l_k = 0, 1, \ldots, n_k - e_k$,
%we can easily get lower and upper bounds for the cdf $F_{C^k_t}(l_k \mid \ldots)$?

Furthermore, higher expected lifetimes for the components must mean
a longer system survival, so in turn the $\ul{F}_{C^k_t}(l_k \mid \ldots)$'s
should give us the lower bound for $P(\Tsys > t\mid\ldots)$,
and the $\ol{F}_{C^k_t}(l_k \mid \ldots)$'s the upper bound.
Note that e.g.\ $\ul{F}_{C^k_t}(l_k \mid \ldots)$ may, however, not correspond to a single parameter pair $(\nkz, \ykz)$.
Also, in the system survival equation ($t > \tnow$)
\begin{linenomath*}
\begin{align} \label{eq:25}
\lefteqn{%
P(\Tsys > t\mid\{\nkz,\ykz, \mbf{t}^k_{e_k;n_k}\}_{k=1}^K)} \hspace*{10ex}\nonumber\\ 
 &= \sum_{l_1=0}^{n_1-e_1} \cdots \sum_{l_K=0}^{n_K-e_K} \Phi(l_1,\ldots,l_K) \prod_{k=1}^K
    P(C^k_t = l_k\mid\nkz,\ykz, \mbf{t}^k_{e_k;n_k})
\end{align}
\end{linenomath*}
we cannot simply plug in all the $\ul{P}(C^k_t = l_k\mid \ldots)$ %and $\ol{P}(C^k_t = l_k\mid \ldots)$
to get $\ul{P}(\Tsys > t\mid\nkz,\ykz, \mbf{t}^k_{e_k;n_k})$, % and $\ol{P}(\Tsys > t\mid\nkz,\ykz, \mbf{t}^k_{e_k;n_k})$,
as the $\ul{P}(C^k_t = l_k\mid \ldots)$ could correspond to different $(\nkz,\ykz)$ for each $l_k$.
(It would give us a lower bound for $\ul{P}(\Tsys > t\mid\ldots)$, however, but it might be very coarse.)

Writing out Equation~\eqref{eq:25}, we get
\begin{linenomath*}
\begin{align}
\lefteqn{P(\Tsys > t\mid\{\nkz,\ykz, \mbf{t}^k_{e_k;n_k}\}_{k=1}^K)} \\
 &= \sum_{l_1=0}^{n_1-e_1} \cdots \sum_{l_K=0}^{n_K-e_K} \Phi(l_1,\ldots,l_K) \prod_{k=1}^K
    P(C^k_t = l_k\mid\nkz,\ykz, \mbf{t}^k_{e_k;n_k}) \\
 &= \sum_{l_1=0}^{n_1-e_1} \cdots \sum_{l_K=0}^{n_K-e_K} \Phi(l_1,\ldots,l_K) \prod_{k=1}^K
    \sum_{j=0}^{n_k-e_k-l_k} (-1)^j \frac{(n_k - e_k)!}{l_k! j! (n_k - e_k - l_k - j)!} \times \\ & \hspace*{45ex}
    \left(\frac{\nkn\ykn}{\nkn\ykn + (l_k + j) (t^\kappa - (\tnow)^\kappa)}\right)^{\nkn + 1} \\
 &= \sum_{l_1=0}^{n_1-e_1} \cdots \sum_{l_K=0}^{n_K-e_K} \Phi(l_1,\ldots,l_K) \prod_{k=1}^K \label{eq:30}
    \sum_{j=0}^{n_k-e_k-l_k} (-1)^j \frac{(n_k - e_k)!}{l_k! j! (n_k - e_k - l_k - j)!} \times \\ & \hspace*{17ex}
    \left(\frac{\nkz\ykz + \sum_{i=1}^{e_k} (t_i^k)^\kappa + (n_k-e_k) (\tnow)^\kappa }%
               {\nkz\ykz + \sum_{i=1}^{e_k} (t_i^k)^\kappa + (n_k-e_k-l_k-j) (\tnow)^\kappa + (l_k + j) t^\kappa }\right)^{%
    \nkz + e_k + 1} 
\end{align}
\end{linenomath*}

While it is clear that the lower bounds for $\ykz$ will lead to the lower system survival function,
the role of $\nkz$ is not so clear.
We therefore resort to numerical optimization of \eqref{eq:25} over $\{\nkz, k=1,\ldots,K\}$
to obtain the lower bound for $P(\Tsys > t\mid\{\nkz,\ykz, \mbf{t}^k_{e_k;n_k}\}_{k=1}^K)$.


\section{elicitation}

*** prior via pseudo-observations, can contain censored pseudo-data 

*** cf paper from Frank 20ys ago (?)




\end{document}
