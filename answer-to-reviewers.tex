%\documentclass[Journal,SectionNumbers,InsideFigs]{ascelike}
\documentclass[Journal,SectionNumbers,SingleSpace,InsideFigs]{ascelike}

%\usepackage{subfigure}
%\usepackage{epsfig}
%\usepackage{timesmt}

\usepackage[utf8]{inputenc}
\usepackage[OT1]{fontenc}
\usepackage{graphicx}
\usepackage[english]{babel}

\usepackage{amsmath}
\usepackage{amsfonts}
\usepackage{amssymb}
\usepackage{amsthm}
\usepackage{bm}

\usepackage[usenames,dvipsnames]{xcolor}
\usepackage{booktabs}
\usepackage{url}
\usepackage{tikz}
%\usetikzlibrary{shapes.misc,fit}
\usetikzlibrary{%
   arrows,%
   calc,%
   fit,%
   patterns,%
   plotmarks,%
   shapes.geometric,%
   shapes.misc,%
   shapes.symbols,%
   shapes.arrows,%
   shapes.callouts,%
   shapes.multipart,%
   shapes.gates.logic.US,%
   shapes.gates.logic.IEC,%
   er,%
   automata,%
   backgrounds,%
   chains,%
   topaths,%
   trees,%
   petri,%
   mindmap,%
   matrix,%
   calendar,%
   folding,%
   fadings,%
   through,%
   patterns,%
   positioning,%
   scopes,%
   decorations.fractals,%
   decorations.shapes,%
   decorations.text,%
   decorations.pathmorphing,%
   decorations.pathreplacing,%
   decorations.footprints,%
   decorations.markings,%
   shadows}

\usepackage{lineno} % comment this out to get line number again
%\usepackage{hyperref}
\usepackage[bookmarks]{hyperref}
%\usepackage[colorlinks=true,citecolor=red,linkcolor=black]{hyperref}

\newcommand{\reals}{\mathbb{R}}
\newcommand{\posreals}{\reals_{>0}}
\newcommand{\posrealszero}{\reals_{\ge 0}}
\newcommand{\naturals}{\mathbb{N}}

\newcommand{\dd}{\,\mathrm{d}}

\newcommand{\mbf}[1]{\mathbf{#1}}
\newcommand{\bs}[1]{\boldsymbol{#1}}
\renewcommand{\vec}[1]{{\bm#1}}

\newcommand{\uz}{^{(0)}} % upper zero
\newcommand{\un}{^{(n)}} % upper n
\newcommand{\ui}{^{(i)}} % upper i

\newcommand{\ul}[1]{\underline{#1}}
\newcommand{\ol}[1]{\overline{#1}}

\newcommand{\Rsys}{R_\text{sys}}
\newcommand{\lRsys}{\ul{R}_\text{sys}}
\newcommand{\uRsys}{\ol{R}_\text{sys}}

\newcommand{\Fsys}{F_\text{sys}}
\newcommand{\lFsys}{\ul{F}_\text{sys}}
\newcommand{\uFsys}{\ol{F}_\text{sys}}

\def\Rsys{R_\text{sys}}
\def\Tsys{T_\text{sys}}

\newcommand{\E}{\operatorname{E}}
\newcommand{\V}{\operatorname{Var}}
\newcommand{\wei}{\operatorname{Wei}} % Weibull Distribution
\newcommand{\ig}{\operatorname{IG}}   % Inverse Gamma Distribution

\newcommand{\El}{\ul{\operatorname{E}}}
\newcommand{\Eu}{\ol{\operatorname{E}}}

\def\yz{y\uz}
\def\yn{y\un}
%\def\yi{y\ui}
\newcommand{\yfun}[1]{y^{({#1})}}
\newcommand{\yfunl}[1]{\ul{y}^{({#1})}}
\newcommand{\yfunu}[1]{\ol{y}^{({#1})}}

\def\ykz{y\uz_k}
\def\ykn{y\un_k}

\def\yzl{\ul{y}\uz}
\def\yzu{\ol{y}\uz}
\def\ynl{\ul{y}\un}
\def\ynu{\ol{y}\un}
\def\yil{\ul{y}\ui}
\def\yiu{\ol{y}\ui}

\def\ykzl{\ul{y}\uz_k}
\def\ykzu{\ol{y}\uz_k}
\def\yknl{\ul{y}\un_k}
\def\yknu{\ol{y}\un_k}

\newcommand{\ykzfun}[1]{y\uz_{#1}}
\newcommand{\ykzlfun}[1]{\ul{y}\uz_{#1}}
\newcommand{\ykzufun}[1]{\ol{y}\uz_{#1}}

\def\nz{n\uz}
\def\nn{n\un}
%\def\ni{n\ui}
\newcommand{\nfun}[1]{n^{({#1})}}
\newcommand{\nfunl}[1]{\ul{n}^{({#1})}}
\newcommand{\nfunu}[1]{\ol{n}^{({#1})}}

\def\nkz{n\uz_k}
\def\nkn{n\un_k}
\newcommand{\nkzfun}[1]{n\uz_{#1}}
\newcommand{\nkzlfun}[1]{\ul{n}\uz_{#1}}
\newcommand{\nkzufun}[1]{\ol{n}\uz_{#1}}

\def\nzl{\ul{n}\uz}
\def\nzu{\ol{n}\uz}
\def\nnl{\ul{n}\un}
\def\nnu{\ol{n}\un}
\def\nil{\ul{n}\ui}
\def\niu{\ol{n}\ui}

\def\nkzl{\ul{n}\uz_k}
\def\nkzu{\ol{n}\uz_k}
\def\nknl{\ul{n}\un_k}
\def\nknu{\ol{n}\un_k}


\def\taut{\tau(\vec{t})}
\def\ttau{\tilde{\tau}}
\def\ttaut{\ttau(\vec{t})}

\def\MZ{\mathcal{M}\uz}
\def\MN{\mathcal{M}\un}

\def\MkZ{\mathcal{M}\uz_k}
\def\MkN{\mathcal{M}\un_k}

\def\PkZ{\Pi\uz_k}
\def\PkN{\Pi\un_k}
\newcommand{\PZi}[1]{\Pi\uz_{#1}}


\def\tnow{t_\text{now}}
\def\tpnow{t^+_\text{now}}

\newcommand{\comments}[1]{{\small\color{gray} #1}}

\newtheorem{example}{Example}

\allowdisplaybreaks


\begin{document}
\title{RESPONSE TO REVIEWERS' COMMENTS TO OUR MANUSCRIPT RUENG-211}

\author{
Gero Walter%
\thanks{
Postdoctoral Researcher,
School of Industrial Engineering,
Eindhoven University of Technology, Eindhoven, The Netherlands.
E-mail: g.m.walter@tue.nl. Corresponding author.},
\ Ph.D.
\\
and
Frank P.A. Coolen%
\thanks{
Professor,
Department of Mathematical Sciences,
Durham University, Durham, United Kingdom.
E-mail: frank.coolen@durham.ac.uk.},
\ Ph.D.
}

\maketitle

\begin{abstract}
We would like to thank the two anonymous reviewers for their detailed and constructive comments,
and their overall positive votum.
We reproduce their comments here verbatim,
and have numbered the reviewers' comments and questions to allow for easy referencing. 
Our answers are interlaced and typeset in \emph{italic}.
\end{abstract}


\section*{Reviewer 1}

The authors presented an approach to mitigate the issue of prior-data conflict.
The idea is to consider sets of prior distributions to model limited knowledge on Weibull distributed component lifetimes,
and then to model the systems with arbitrary layout using the survival signature.
As mentioned by the authors the calculation of lower and upper predictive system reliability bounds is tractable.

\smallskip

\emph{***Drop notion of `mitigate', replace with `highlight' / etc? (appears in abstract and end of introduction only) ***\\
We present a method that explicitly shows, through imprecision in the inferences, when prior and data vary substantially.}

The manuscript is well written, clear and complete. The manuscript is ready to be published.
However, the authors might want to consider to comment about:
\begin{enumerate}
\item how this approach performs compared with existing and traditional approaches

\smallskip

\emph{We discuss our changes related to existing and traditional approaches in the answers to Reviewer 2's comments below.}

\item are any numerical approaches that can be used to perform system reliability based on the survival signature
or consider arbitrary distributions for the component failure time?

\smallskip

\emph{We are not sure if we have understood this comment well.
The survival signature allows to calculate the system reliability for arbitrary component lifetime distributions,
as is clear from Equation~(13), this is explained in detail in the cited reference (Coolen and Coolen-Maturi 2012).
We have added a sentence to emphasize this fact.}
\end{enumerate}

Furthermore, at page 15 the authors make a statement that this method has been implemented in R
and it is intended to be released as a package. Is the code already available? 

\smallskip

\emph{Yes, it was our intention to provide an \textsf{R} package when we submitted the manuscript.
A package for calculating survival signatures called \texttt{ReliabilityTheory} is available,
but for other computations, we were not able to provide a package yet. %Unfortunately, the first author's ***
However, well-commented code for all computations and figures is available upon request,
and can also be found in the github repository
\texttt{http://www.github.com/geeeero/asce-asme2016}.***}


\section*{Reviewer 2}

The paper is well written and interesting enough.
Nonetheless, the products of research are not really new in the fields of Bayesian analysis and reliability analysis.
In my view, there is a deep lack of bibliographical review,
and the subject is not sufficiently argued by a detailed case-study
(typically, there is no real expert information for the case-study considered, and the authors make purely subjective assumptions).
The Weibull distribution (which is in fact simplified to the Weibull one) is clearly very usual,
and the conjugate priors and virtual data are usual tools.
The authors do not detail formally what is a prior-data conflict, while several definitions exist.

\smallskip

\emph{There are several issues here:}
\begin{enumerate}
\item \emph{Contrary to the reviewer's suggestion, the work presented in this paper is novel.
First, prior-data conflict within imprecise probability has not yet been presented within a reliability context.
Secondly, the combination of imprecise probability methods that can reflect prior-data conflict with the survival signature,
enabling far more large-scale applications than without the use of the survival signature, is novel.}
\item \emph{Deep lack of bibliographic review:
there is very little work thus far on imprecise probability methods for prior-data conflict, indeed we have referred to it.
However, as is also clear from the reviewer's later comments,
the suggestion is probably to extend on literature for prior-data conflict in PRECISE Bayesian analysis.
The paper already contained a clear argument on why precise probabilistic (Bayesian) methods struggle to reflect,
and deal with, prior-data conflict, while imprecise methods, as shown, can do so in an intuitively attractive way,
namely by increasing imprecision for an event of interest.
Nevertheless, the reviewer has pointed us to some interesting papers, for which we are grateful.
We have included a brief mentioning of two of the suggested papers,
Bousquet (2008) and O'Hagan and Perrichi (2012), in the introductory section.}
\item \emph{Detail (or define?) what a prior-data conflict is:
A formal definition of prior-data conflict in the imprecise probability context is given in Definition~6
of Walter and Augustin (2009), which we cite in both the Introduction and in Section~3.
Crucially, our approach is different from precise Bayesian methods for identifying prior-data conflict (or agreement).
As we explain in Section~3,
the magnitude of the set of posteriors, as compared to the magnitude of the set of priors,
reflects the degree of prior-data conflict.
Prior-data conflict is present when imprecision increases,
and we think this is a far more attractive notion than any of the many possible definitions in precise Bayesian approaches.
We have included a short discussion about possible further research into measuring prior-data conflict
based on imprecise probabilistic inference.}
\end{enumerate}

The main advantage of this article is that it clearly presents Bayesian robustness (except the decision-theoretical part,
which is a bit prejudicial), which can be valuable for engineers.
Therefore the authors should provide, in my view, a more detailed presentation of prior-data conflicts,
provide more details about the reality of their case-study and go to the final applicability of their methodology.
A very good point is the production of a software package, but no detail is given about the name and the availability.

\smallskip

\emph{We do not really understand the `except' bit of the first sentence.
The main issue here is that we do NOT present a case study, and indeed make this already very clear in the text.
We mention explicitly `examples illustrating' at the end of Section~1,
and at the start of Section~6, and at the start of Section~6.1 it is clear that these are not really elicited expert opinions.
On software, please see our comment to Reviewer~1's last question.}

\smallskip

Here are some comments on the text:

\begin{enumerate}
\item Abstract: The term "straightforward" is to be relaxed, since prior elicitation is clearly not easy to conduct in practice.
This sentence is contradictory with l.~52, for instance.
I suspect that the authors are too optimistic since the Weibull case with a known shape parameter is very simplistic.

\smallskip
\emph{Yes, we agree that the inclusion of expert knowledege is straightforward only in the mathematical sense,
and have replaced `straightforward' in line 3 of the abstract by `possible'.}

\smallskip
\item L.~32: Could you define formally what is a ``survival signature''?

\smallskip
\emph{The survival signature is explained and defined in Section~4.2, ``System Reliability using the Survival Signature'',
more details are given in the reference Coolen and Coolen-Maturi (2012) cited in line~32 and in Section~4.2.
As mentioned in our answers to Reviewer 1's comments,
we have added a sentence to Section~4.2 highlighing the usefulness of the survival signature.}

\smallskip
\item L.~40 and following. Actually, up to a given parameterisation, all models are exponential since the shape parameters are known.
According to my experience, this situation appears scarcely.
This is the first reason why I have doubts on the reality of the motivating case-study (this motivation could besides appear within the Abtract).
The authors could have a fair description of their model by noticing it.

\smallskip
\emph{Again, our paper is not a case study.
The example is chosen to highlight the approach through a simple model, that can be widely understood.
It being simple does not mean it is not relevant!
Of course, more complicated models will be of interest for later research building on this,
in particular towards larger real world applications,
but the system failure time distribution here is already quite complex as we only make the model assumption at component level.}

\smallskip
\item P.~6: It is more traditional to use the Gamma prior for 1/lambda rather than the inverse Gamma prior on lambda.
The reason is that the virtual size $\nz$ can be in fact equal to $a$ (and not $a-1$),
since it should be possible to have $a<1$ and get a proper prior
(typically, an expert information could be considered as providing as much information as 1/2 real observation).
I am not convinced by the choice $\nz=a-1$.
Besides, a clearer interpretation of the conjugate prior (for all models allowing conjugate priors)
is the virtual data posterior prior (using an original noninformative Jeffreys prior). See [1] for details and reference therein.

\smallskip
\emph{Our parametrisation is chosen in order to facilitate the novel prior-data conflict method in the imprecise probabilistic approach.
We can allow a very small weight of the expert judgements but this would not illustrate (strong) prior data conflict well:
in order to have prior-data conflict one needs pretty informative priors.
We do not expand on the non-informative case as that is explicitly not of interest here.}

\smallskip
\item P.~7: I understand the comparison made by the authors about the prior-data conflict.
But it can be explained uniquely in terms of posterior sensitivity to a very small amount of virtual (hypothetical) or real data.
In fact, this is a bit unfair in my view to examine the prior impact alone,
then the prior and data impact through the posterior, for data which are not sampled by fixed ....as other prior guesses.
I think that the authors should read more deeply the literature dedicated to prior-data conflict
(not only the paper by Evans \& Moshonov, but others [2,3]) and summarize their conclusions:
the conflict can be detected on the sample space, or in the parameter space.
In the first solution (chosen by Evans \& Moshonov), a conflict can be not detected when the dimension increases,
because of the possible presence of ancillary statistics.
Since the authors are working only on a very simplified class of models,
for which the statistics are exhaustive, this problem remains underlying.
Furthermore, a large part of the literature is dedicated to create priors
that lead to robust conclusion in the sense a conflict is removed [4].
In [3], the symmetric notion of prior-data agreement examines the case of non-conjugate Weibull models,
and it could give ideas to the authors for improving this part of the article
(and possibly the conclusion about giving more flexibility to priors;
it seems that the virtual size $\nz$ could be a useful calibration lever, to modulate for removing a possible conflict).

\smallskip
\emph{This all comes down again to what one can do within the precise Bayesian framework,
which remains limited---the authors have certainly read the literature well (see the thesis Walter (2013), e.g.),
but the reviewer seems to miss the key point:
the imprecise framework deals fundamentally different with information and prior-data conflict,
and we claim and illustrate that this has advantages.
Note also that ``modulating $\nz$'' is not allowed in a careful (precise) Bayesian analysis:
it would make the prior dependent on the data and as such go against the fundamental principles,
although of course it is regularly done (e.g. `empirical Bayes') and it seems that researchers get away with this.
It must be emphasized that the method we present does not make the set of priors dependent on the data
and as such is fundamentally sound from the (robust) Bayesian perspective.}

\smallskip
\item P.~10 and Figure~2: It could be interesting to see (graphically) the difference between the approach of the authors
with a hierarchically Bayesian approach using uniform distributions between the bounds.
I am not sure that the difference are really high.

\smallskip
\emph{We do not think it would be that interesting to see any precise distribution in this plot, as it is all about the imprecision.}

\smallskip
\item L.~270: Place "Equations" before (19) and (20).

\smallskip
\emph{Yes, thanks, we have done so now.}

\smallskip
\item L.~293: Of which package do you talk? Other precisions from the submission?

\smallskip
\emph{Please see our comment to Reviewer~1's last question. We have changed the wording at the end of Section~4 accordingly.}

\smallskip
\item L.~342: ``One can imagine'' is a bit surprising,
since it tends to show that the considered case-study was not really examined using expert sollicitation.
Could the authors give more détails about the questioning,
or the assumptions made to admit that the results could very likely be arising from real experts?

\smallskip
\emph{As commented above, we do not present a case study here, and have expressed that very clearly before.}

\smallskip
\item Section~7: At the image of the full paper, this section is well written and interesting.
But I think that the authors should be more precise when they talk about the (more reasonable) case of two-parameter Weibull distributions.
There is a huge work on non-conjugate priors, and I think that the authors could refer to the work evoked hereinbefore,
among others, to discuss the nature of what could be a good prior for such models.

\smallskip
\emph{As we point out in Section~7, the paper should be seen as a starting point for more complex and intricate modelling and analysis,
and that further research would be needed, where the usually used discrete priors for the shape parameter would be a possible starting point.
However, a more detailed discussion on priors suitable for an imprecise probability approach is out of scope in our view,
since the requirements for an imprecise modelling approach depart markedly from those in a precise Bayesian approach.}
\end{enumerate}

\end{document}
