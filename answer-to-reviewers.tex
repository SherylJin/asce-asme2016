%\documentclass[Journal,SectionNumbers,InsideFigs]{ascelike}
\documentclass[Journal,SectionNumbers,SingleSpace,InsideFigs]{ascelike}

%\usepackage{subfigure}
%\usepackage{epsfig}
%\usepackage{timesmt}

\usepackage[utf8]{inputenc}
\usepackage[OT1]{fontenc}
\usepackage{graphicx}
\usepackage[english]{babel}

\usepackage{amsmath}
\usepackage{amsfonts}
\usepackage{amssymb}
\usepackage{amsthm}
\usepackage{bm}

\usepackage[usenames,dvipsnames]{xcolor}
\usepackage{booktabs}
\usepackage{url}
\usepackage{tikz}
%\usetikzlibrary{shapes.misc,fit}
\usetikzlibrary{%
   arrows,%
   calc,%
   fit,%
   patterns,%
   plotmarks,%
   shapes.geometric,%
   shapes.misc,%
   shapes.symbols,%
   shapes.arrows,%
   shapes.callouts,%
   shapes.multipart,%
   shapes.gates.logic.US,%
   shapes.gates.logic.IEC,%
   er,%
   automata,%
   backgrounds,%
   chains,%
   topaths,%
   trees,%
   petri,%
   mindmap,%
   matrix,%
   calendar,%
   folding,%
   fadings,%
   through,%
   patterns,%
   positioning,%
   scopes,%
   decorations.fractals,%
   decorations.shapes,%
   decorations.text,%
   decorations.pathmorphing,%
   decorations.pathreplacing,%
   decorations.footprints,%
   decorations.markings,%
   shadows}

\usepackage{lineno} % comment this out to get line number again
%\usepackage{hyperref}
\usepackage[bookmarks]{hyperref}
%\usepackage[colorlinks=true,citecolor=red,linkcolor=black]{hyperref}

\newcommand{\reals}{\mathbb{R}}
\newcommand{\posreals}{\reals_{>0}}
\newcommand{\posrealszero}{\reals_{\ge 0}}
\newcommand{\naturals}{\mathbb{N}}

\newcommand{\dd}{\,\mathrm{d}}

\newcommand{\mbf}[1]{\mathbf{#1}}
\newcommand{\bs}[1]{\boldsymbol{#1}}
\renewcommand{\vec}[1]{{\bm#1}}

\newcommand{\uz}{^{(0)}} % upper zero
\newcommand{\un}{^{(n)}} % upper n
\newcommand{\ui}{^{(i)}} % upper i

\newcommand{\ul}[1]{\underline{#1}}
\newcommand{\ol}[1]{\overline{#1}}

\newcommand{\Rsys}{R_\text{sys}}
\newcommand{\lRsys}{\ul{R}_\text{sys}}
\newcommand{\uRsys}{\ol{R}_\text{sys}}

\newcommand{\Fsys}{F_\text{sys}}
\newcommand{\lFsys}{\ul{F}_\text{sys}}
\newcommand{\uFsys}{\ol{F}_\text{sys}}

\def\Rsys{R_\text{sys}}
\def\Tsys{T_\text{sys}}

\newcommand{\E}{\operatorname{E}}
\newcommand{\V}{\operatorname{Var}}
\newcommand{\wei}{\operatorname{Wei}} % Weibull Distribution
\newcommand{\ig}{\operatorname{IG}}   % Inverse Gamma Distribution

\newcommand{\El}{\ul{\operatorname{E}}}
\newcommand{\Eu}{\ol{\operatorname{E}}}

\def\yz{y\uz}
\def\yn{y\un}
%\def\yi{y\ui}
\newcommand{\yfun}[1]{y^{({#1})}}
\newcommand{\yfunl}[1]{\ul{y}^{({#1})}}
\newcommand{\yfunu}[1]{\ol{y}^{({#1})}}

\def\ykz{y\uz_k}
\def\ykn{y\un_k}

\def\yzl{\ul{y}\uz}
\def\yzu{\ol{y}\uz}
\def\ynl{\ul{y}\un}
\def\ynu{\ol{y}\un}
\def\yil{\ul{y}\ui}
\def\yiu{\ol{y}\ui}

\def\ykzl{\ul{y}\uz_k}
\def\ykzu{\ol{y}\uz_k}
\def\yknl{\ul{y}\un_k}
\def\yknu{\ol{y}\un_k}

\newcommand{\ykzfun}[1]{y\uz_{#1}}
\newcommand{\ykzlfun}[1]{\ul{y}\uz_{#1}}
\newcommand{\ykzufun}[1]{\ol{y}\uz_{#1}}

\def\nz{n\uz}
\def\nn{n\un}
%\def\ni{n\ui}
\newcommand{\nfun}[1]{n^{({#1})}}
\newcommand{\nfunl}[1]{\ul{n}^{({#1})}}
\newcommand{\nfunu}[1]{\ol{n}^{({#1})}}

\def\nkz{n\uz_k}
\def\nkn{n\un_k}
\newcommand{\nkzfun}[1]{n\uz_{#1}}
\newcommand{\nkzlfun}[1]{\ul{n}\uz_{#1}}
\newcommand{\nkzufun}[1]{\ol{n}\uz_{#1}}

\def\nzl{\ul{n}\uz}
\def\nzu{\ol{n}\uz}
\def\nnl{\ul{n}\un}
\def\nnu{\ol{n}\un}
\def\nil{\ul{n}\ui}
\def\niu{\ol{n}\ui}

\def\nkzl{\ul{n}\uz_k}
\def\nkzu{\ol{n}\uz_k}
\def\nknl{\ul{n}\un_k}
\def\nknu{\ol{n}\un_k}


\def\taut{\tau(\vec{t})}
\def\ttau{\tilde{\tau}}
\def\ttaut{\ttau(\vec{t})}

\def\MZ{\mathcal{M}\uz}
\def\MN{\mathcal{M}\un}

\def\MkZ{\mathcal{M}\uz_k}
\def\MkN{\mathcal{M}\un_k}

\def\PkZ{\Pi\uz_k}
\def\PkN{\Pi\un_k}
\newcommand{\PZi}[1]{\Pi\uz_{#1}}


\def\tnow{t_\text{now}}
\def\tpnow{t^+_\text{now}}

\newcommand{\comments}[1]{{\small\color{gray} #1}}

\newtheorem{example}{Example}

\allowdisplaybreaks


\begin{document}
\title{RESPONSE TO REVIEWERS' COMMENTS TO OUR MANUSCRIPT RUENG-211}

\author{
Gero Walter%
\thanks{
Postdoctoral Researcher,
School of Industrial Engineering,
Eindhoven University of Technology, Eindhoven, The Netherlands.
E-mail: g.m.walter@tue.nl. Corresponding author.},
\ Ph.D.
\\
and
Frank P.A. Coolen%
\thanks{
Professor,
Department of Mathematical Sciences,
Durham University, Durham, United Kingdom.
E-mail: frank.coolen@durham.ac.uk.},
\ Ph.D.
}

\maketitle

\begin{abstract}
We would like to thank the two anonymous reviewers for their detailed and constructive comments,
and their overall positive votum.
We reproduce their comments here verbatim,
and have numbered the reviewers' comments and questions to allow for easy referencing. 
Our answers are interlaced and typeset in \emph{italic}.
\end{abstract}


\section*{Reviewer 1}

The authors presented an approach to mitigate the issue of prior-data conflict.
The idea is to consider sets of prior distributions to model limited knowledge on Weibull distributed component lifetimes,
and then to model the systems with arbitrary layout using the survival signature.
As mentioned by the authors the calculation of lower and upper predictive system reliability bounds is tractable.

\smallskip

\emph{***Drop notion of `mitigate', replace with `highlight' / etc? (appears in abstract and end of introduction only) ***\\
We present a method that explicitly shows, through imprecision in the inferences, when prior and data vary substantially.}

The manuscript is well written, clear and complete. The manuscript is ready to be published.
However, the authors might want to consider to comment about:
\begin{enumerate}
\item how this approach performs compared with existing and traditional approaches

\smallskip

\emph{We discuss our changes related to existing and traditional approaches in the answers to Reviewer 2's comments below.}

\item are any numerical approaches that can be used to perform system reliability based on the survival signature
or consider arbitrary distributions for the component failure time?

\smallskip

\emph{We are not sure if we have understood this comment well.
The survival signature allows to calculate the system reliability for arbitrary component lifetime distributions,
as is clear from Equation~(13), this is explained in detail in the cited reference (Coolen and Coolen-Maturi 2012).
We have added a sentence to emphasize this fact.}
\end{enumerate}

Furthermore, at page 15 the authors make a statement that this method has been implemented in R
and it is intended to be released as a package. Is the code already available? 

\smallskip

\emph{Yes, it was our intention to provide an \textsf{R} package when we submitted the manuscript.
A package for calculating survival signatures called \texttt{ReliabilityTheory} is available,
but for other computations, we were not able to provide a package yet. %Unfortunately, the first author's ***
However, well-commented code for all computations and fugures is available upon request,
and is also in the github repository
\texttt{http://www.github.com/geeeero/asce-asme2016}.***}


\section*{Reviewer 2}

The paper is well written and interesting enough.
Nonetheless, the products of research are not really new in the fields of Bayesian analysis and reliability analysis.
In my view, there is a deep lack of bibliographical review,
and the subject is not sufficiently argued by a detailed case-study
(typically, there is no real expert information for the case-study considered, and the authors make purely subjective assumptions).
The Weibull distribution (which is in fact simplified to the Weibull one) is clearly very usual,
and the conjugate priors and virtual data are usual tools.
The authors do not detail formally what is a prior-data conflict, while several definitions exist.

\smallskip

\emph{There are several issues here:}
\begin{enumerate}
\item \emph{Contrary to the reviewer's suggestion, the work presented in this paper is novel.
First, prior-data conflict within imprecise probability has not yet been presented within a reliability context.
Secondly, the combination of imprecise probability methods that can reflect prior data conflict with the survival signature,
enabling far more large-scale applications than without the use of the survival signature, is novel.}
\item \emph{Deep lack of bibliographic review:
there is very little work thus far on imprecise probability methods for prior data conflict, indeed we have referred to it.
However, as is also clear from the reviewer's later comments,
the suggestion is probably to extend on literature for prior data conflict in PRECISE Bayesian analysis.
The paper already contained a clear argument on why precise probabilistic (Bayesian) methods struggle to reflect,
and deal with, prior-data conflict, while imprecise methods, as shown, can do so in an intuitively attractive way,
namely by increasing imprecision for an event of interest.
Nevertheless, the reviewer has pointed us to some interesting papers, for which we are grateful.
We have included a brief mentioning of two of the suggested papers,
Bousquet (2008) and O'Hagan and Perrichi (2012), in the introductory section.}
\item \emph{Detail (or define?) what a prior-data conflict is:
A formal definition of prior-data conflict for our imprecise probability approach is given in Walter and Augustin (2009, Definition~6). 
Our approach departs from precise Bayesian methods for identifying prior-data conflict (or agreement)
by *** ***
Frank: this is a somewhat interesting point.
In our case, it effectively is SHOWN by the method when increased imprecision occurs
which I think is far more attractive than one of the many possible `definitions'. But perhaps adding a sentence to the end section (as suggested just above) about possible further research into this may be good.}
\end{enumerate}



\end{document}
